\documentclass[12pt]{article}
\usepackage[russian]{babel}
\usepackage[utf8x]{inputenc}
\usepackage{indentfirst}
\usepackage{amsmath}
\usepackage{fixltx2e}
\usepackage{vmargin}
\usepackage{graphicx}
\linespread{1.3}
\renewcommand{\rmdefault}{ftm}
\graphicspath{ {./images/} }
\setmarginsrb{30mm}{20mm}{15mm}{20mm}{0pt}{0pt}{0pt}{13mm}
\title{This is the END}
\author{This is me}
\date{ 30 сегодня 2019}
\begin{document}
\newtheorem{thmv}{Theorem}
\maketitle
\section{The first one}
Cегодня мы рассмотрим популяционную модель Мэя с неограниченным распределенным запаздыванием\\
Модель относится к семейству моделей <<хищник-жертва>>\\
Допускается, что все параметры системы положительны
\subsection{Популяционная модель Мэя}
\begin{equation*}
    \begin{cases}
    N(t)'=r N(t)(1-\int_{-\infty}^{0} Q(-s)N(t+s) e^s ds - \alpha P(t) N(t))\\ 
    P(t)'=\frac{d P(t)}{d t}= P(t)N(t)
\end{cases}
\end{equation*}
   \begin{thmv}
  Бифуркация Хопфа возможна при выполнении следующих условий:
    \[1)  \lambda=\pm i \nu \]
    \[2)  \lambda=\pm i \nu + \lambda \textsubscript{1} \mu + O(\mu \textsuperscript{2}) \] \text{где Re $\lambda \textsubscript{1} \neq$ 0}\ 
     \end{thmv}
    \begin{center}
        Текст ВКР должен быть оформлен по следующим правилам:\\
        \Large times new roman 12-14pt
    \end{center}
\begin{equation}
\label{math/1}
    y_1^1 (t)=e^{it}+e^{-it}
\end{equation}
Это уравнение (\ref{math/1}) является решением \emph{ первого приближения }
  \begin{multline}
      y(t,\gamma)=\sum_{k=1}^n \gamma^k y^k (t)= \gamma y(t)+\gamma^2 y^2(t)+\gamma^3 y^3(t)+\gamma^4 y^4 (t)  + \gamma^5 y^5(t)+\gamma^6 y^6(t) \\+\gamma^7 y^7(t) +\gamma^8 y^8(t)+\gamma^9 y^9(t)
 \label{math/2}
  \end{multline}
  \begin{equation*}
            f_1^2(\tau)=d_1^1 e^{i \nu \tau}+d_1^{-1} e^{-i \nu \tau} +d_1^2e^{2 i \nu \tau}+d_1^{-2}e^{-2 i \nu \tau}
  \end{equation*}
  \begin{table}
      \centering
      \begin{tabular}{ | c c |}
         \hline
           5 & 10 \\
          11 & 12 \\
          \hline
      \end{tabular}
      \caption{ Коэффициенты разложения функции в ряд Тейлора}
      \label{tab 1l}
  \end{table}
  \[ \Bigg(
  \begin{matrix}
        1 & -i+\gamma_1 & 0 \\
        0 & 1 & -1 + \gamma_1 \\
        \gamma_1 & 1 & -i
  \end{matrix}
   \Bigg)\]
\newpage
\begin{figure}
    \centering
    \includegraphics[scale=0.2, angle=45]{nhl}
    \caption{it is a good day to ...}
    \label{picture1}
\end{figure}
На рисунке \ref{picture1} представлена эмблема клуба веселых и находчивых математиков из Лас-Вегаса.\\
\Large Формула \ref{math/2} представляет собой степенной ряд нашего решения в разложении по вспомогательному параметру $\gamma $ \\
\emph{ В таблице (\ref{tab 1l}) указаны коэффициенты разложения функции в ряд Тейлора}\\
\underline{Информацию о модели Мея вы можете найти на \cite{Dolgiy}[стр.~115].}\\
\textbf{ Формулу (\ref{math/1}) можно встретить в таких учебных пособиях как \cite{Pimenov} и \cite{landau}.}
\newpage
\begin{thebibliography}{5}
    \bibitem{Dolgiy}
    Ю.Ф. Долгий, П.Г.Сурков. Математические модели динамических систем с запаздыванием. Екатеринбург. Издательство Уральского университета.2012г.
    \bibitem{Pimenov}
    В.Г. Пименов. Численные методы. Екатеринбург. Издательство Уральского университета, 2012г.
    \bibitem{landau}
    Л.Д.Ландау, Е.М.Лифшиц. Статистическая физика.Москва, 1969г.
    \end{thebibliography}
 \end{document}

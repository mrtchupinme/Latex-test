\documentclass[12pt]{article}
\usepackage[latin,russian]{babel}
\usepackage[utf8x]{inputenc}
\usepackage{indentfirst}
\usepackage{amsmath}
\usepackage{fixltx2e}
\title{This is the END}
\author{This is me}
\begin{document}
\newtheorem{thmv}{Theorem}
\maketitle
\section{The first one}
Cегодня мы рассмотрим популяционную модель Мэя с неограниченным распределенным запаздыванием\\
Модель относится к семейству моделей <<хищник-жертва>>\\
Допускается, что все параметры системы положительны
\subsection{Популяционная модель Мэя}
\begin{equation*}
    \begin{cases}
    N(t)'=r N(t)(1-\int_{-\infty}^{0} Q(-s)N(t+s) e^s ds - \alpha P(t) N(t))\\ 
    P(t)'=\frac{d P(t)}{d t}= P(t)N(t)
\end{cases}
\end{equation*}
   \begin{thmv}
  Бифуркация Хопфа возможна при выполнении следующих условий:
    \[1)  \lambda=\pm i \nu \]
    \[2)  \lambda=\pm i \nu + \lambda \textsubscript{1} \mu + O(\mu \textsuperscript{2}) \]&\text{где Re $\lambda \textsubscript{1} \neq$ 0}\ 
     \end{thmv}
    \begin{center}
        Текст ВКР должен быть оформлен по следующим правилам:\\
        \Large times new roman 12-14pt
    \end{center}
\begin{equation}
\label{math/1}
    y_1^1 (t)=e^{it}+e^{-it}
\end{equation}
Это уравнение (\ref{math/1}) является решением \emph{ первого приближения }
  \begin{multline}
      y(t,\gamma)=\sum_{k=1}^n \gamma^k y^k (t)= \gamma y(t)+\gamma^2 y^2(t)+\gamma^3 y^3(t)+\gamma^4 y^4 (t)  + \gamma^5 y^5(t)+\gamma^6 y^6(t) \\+\gamma^7 y^7(t) +\gamma^8 y^8(t)+\gamma^9 y^9(t)
  \end{multline}
  \begin{table}
      \centering
      \begin{tabular}{ | c c |}
         \hline
           5 & 10 \\
          11 & 12 \\
          \hline
      \end{tabular}
      \caption{Simple table}
      \label{tab:my_label}
  \end{table}
   \end{document}

\documentclass[12pt]{article}
\usepackage[russian]{babel}
\usepackage[utf8x]{inputenc}
\usepackage{indentfirst}
\usepackage{amsmath}
\usepackage{fixltx2e}
\usepackage{vmargin}
\usepackage{graphicx}
\graphicspath{ {./images/} }
\setmarginsrb{30mm}{20mm}{15mm}{20mm}{0pt}{0pt}{0pt}{13mm}
\begin{document}
\begin{titlepage}
{\small
\centerline{Министерство образования и науки Российской Федерации}
\centerline{Федеральное государственное автономное образовательное учреждение}
\centerline{высшего образования}
\centerline{<<Уральский федеральный университет имени первого президента России Б.Н.Ельцина>>}
\vskip1cm
\centerline{Институт естественных наук и математики}
\centerline{Департамент прикладной математики и механики}
}
\vskip1cm
\null\hfill
\begin{minipage}{0.6\textwidth}
ДОПУСТИТЬ К ЗАЩИТЕ В ГЭК\\
РОП\hfill \rule[-1pt]{4.5cm}{0.4pt}
$\underset{\text{(подпись)}}{\underline{\hspace{3cm}}}$
\hfill
$\underset{\text{(Ф.И.О.}}{\underline{\hspace{4.5cm}}}$\\
\hfill <<\rule[-1pt]{0.5cm}{0.4pt}>>\rule[-1pt]{4cm}{0.4pt} 2020г.
\end{minipage}\\
\vskip1cm
\centerline{\textbf{МАГИСТЕРСКАЯ ДИССЕРТАЦИЯ}}
\centerline{БИФУРКАЦИЯ ХОПФА В ПОПУЛЯЦИОННОЙ МОДЕЛИ МЭЯ}
\centerline{КАЧЕСТВЕННОЕ ИССЛЕДОВАНИЕ ВЛИЯНИЕ ПАРАМЕТРОВ}
\centerline{НА БИФУРКАЦИЮ}
\vskip3.5cm
\noindent
Научный руководитель: д.ф.м.н., профессор Ю.Ф. Долгий\hfill $\underset{\text{(подпись)}}{\underline{\hspace{3cm}}}$\\
Консультант: В.В. Иванов  \hfill $\underset{\text{(подпись)}}{\underline{\hspace{3cm}}}$\\
Консультант: В.В. Иванов  \hfill $\underset{\text{(подпись)}}{\underline{\hspace{3cm}}}$\\
Нормоконтролер: И.И. Иванов \hfill $\underset{\text{(подпись)}}{\underline{\hspace{3cm}}}$\\
Студент группы \rule[-1pt]{1.5cm}{0.4pt}  И.А. Чупин  \hfill $\underset{\text{(подпись)}}{\underline{\hspace{3cm}}}$\\
\vfill
\centerline{Екатеринбург}
\centerline{2020}
\end{titlepage}
\section*{РЕФЕРАТ}
\thispagestyle{empty}   
В данной работе мы рассмотрим бла бла бла...

It this work i will tell you about blah blah blah

\cleardoublepage                     
\pagenumbering{gobble}         

\tableofcontents

\cleardoublepage                   
\pagenumbering{arabic}         

\setcounter{page}{4}  

\section*{ВВЕДЕНИЕ}
\addcontentsline{toc}{section}{ВВЕДЕНИЕ}

В данной статье рассматриваем модель Мэя из \cite{Dolgiy} 
\newpage
\section*{ОСНОВНАЯ ЧАСТЬ}
\addcontentsline{toc}{section}{ОСНОВНАЯ ЧАСТЬ}
\section{Обзор литературы}
в данной работе используется \cite{Pimenov}.
\section{Постановка задачи}
Была поставлена следующая задача..
\section{Описание решения поставленной задачи}
\subsection{Способы и методы аналитического решения задачи}
рассмотрены различные постановки задачи и бла бла бла

\subsection{Методика эксперимента}
Был произведен следующий алгоритм эксперимента
\begin{enumerate}
\item решение частного случая коэффициентов
\item обобщение решения до общего случая коэффициентов
\end{enumerate}
\subsection{Методика проведения и наблюдения эксперимента}
Постановка научной проблемы:
\begin{enumerate}
\item обнаружение недостаточности знаний, противоречий
\item осознание потребности
\item формулирование проблемы
\end{enumerate}
\subsection{Способы и методы решения задачи}
бла бла бла бла бла бла бла бла бла бла
\begin{multline}
      y(t,\gamma)=\sum_{k=1}^n \gamma^k y^k (t)= \gamma y(t)+\gamma^2 y^2(t)+\gamma^3 y^3(t)+\gamma^4 y^4 (t)  + \gamma^5 y^5(t)+\gamma^6 y^6(t) \\+\gamma^7 y^7(t) +\gamma^8 y^8(t)+\gamma^9 y^9(t)
 \label{math/2}
  \end{multline}
\subsection{Результаты и их обсуждение}

просто много бла бла бла
\section*{ЗАКЛЮЧЕНИЕ}
\addcontentsline{toc}{section}{ЗАКЛЮЧЕНИЕ}

В данной работе расмотрена проблема бла бла бла

\begin{thebibliography}{5}
    \bibitem{Dolgiy}
    Ю.Ф. Долгий, П.Г.Сурков. Математические модели динамических систем с запаздыванием. Екатеринбург. Издательство Уральского университета.2012г.
    \bibitem{Pimenov}
    В.Г. Пименов. Численные методы. Екатеринбург. Издательство Уральского университета, 2012г.
    \bibitem{landau}
    Л.Д.Ландау, Е.М.Лифшиц. Статистическая физика.Москва, 1969г.
    \end{thebibliography}
\addcontentsline{toc}{section}{СПИСОК ЛИТЕРАТУРЫ}

\appendix

\section*{ПРИЛОЖЕНИЯ}
\addcontentsline{toc}{section}{ПРИЛОЖЕНИЯ}
\end{document}
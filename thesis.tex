\documentclass[12pt]{article}
\usepackage[russian]{babel}
\usepackage[utf8x]{inputenc}
\usepackage{indentfirst}
\usepackage{amsmath}
\usepackage{fixltx2e}
\usepackage{vmargin}
\usepackage{graphicx}
\graphicspath{ {./images/} }
\setmarginsrb{30mm}{20mm}{15mm}{20mm}{0pt}{0pt}{0pt}{13mm}
\begin{document}
\numberwithin{equation}{section}
\begin{titlepage}
{\small
\centerline{Министерство образования и науки Российской Федерации}
\centerline{Федеральное государственное автономное образовательное учреждение}
\centerline{высшего образования}
\centerline{<<Уральский федеральный университет имени первого президента России Б.Н.Ельцина>>}
\vskip1cm
\centerline{Институт естественных наук и математики}
\centerline{Департамент прикладной математики и механики}
}
\vskip1cm
\null\hfill
\begin{minipage}{0.6\textwidth}
ДОПУСТИТЬ К ЗАЩИТЕ В ГЭК\\
РОП\hfill \rule[-1pt]{4.5cm}{0.4pt}
$\underset{\text{(подпись)}}{\underline{\hspace{3cm}}}$
\hfill
$\underset{\text{(Ф.И.О.}}{\underline{\hspace{4.5cm}}}$\\
\hfill <<\rule[-1pt]{0.5cm}{0.4pt}>>\rule[-1pt]{4cm}{0.4pt} 2020г.
\end{minipage}\\
\vskip1cm
\centerline{\textbf{МАГИСТЕРСКАЯ ДИССЕРТАЦИЯ}}
\centerline{БИФУРКАЦИЯ ХОПФА В ПОПУЛЯЦИОННОЙ МОДЕЛИ МЭЯ}
\centerline{КАЧЕСТВЕННОЕ ИССЛЕДОВАНИЕ ВЛИЯНИЕ ПАРАМЕТРОВ}
\centerline{НА БИФУРКАЦИЮ}
\vskip3.5cm
\noindent
Научный руководитель: д.ф.м.н., профессор Ю.Ф. Долгий\hfill $\underset{\text{(подпись)}}{\underline{\hspace{3cm}}}$\\
Консультант: В.В. Иванов  \hfill $\underset{\text{(подпись)}}{\underline{\hspace{3cm}}}$\\
Консультант: В.В. Иванов  \hfill $\underset{\text{(подпись)}}{\underline{\hspace{3cm}}}$\\
Нормоконтролер: И.И. Иванов \hfill $\underset{\text{(подпись)}}{\underline{\hspace{3cm}}}$\\
Студент группы \rule[-1pt]{1.5cm}{0.4pt}  И.А. Чупин  \hfill $\underset{\text{(подпись)}}{\underline{\hspace{3cm}}}$\\
\vfill
\centerline{Екатеринбург}
\centerline{2020}
\end{titlepage}
\section*{РЕФЕРАТ}
\thispagestyle{empty}   
Чупин Илья Алексеевич. Бифуркация Хопфа в популяционной модели Мэя

Ключевые слова: бифуркация, популяционная модель Мэя, периодическое решение, устойчивость периодического решения.
Работа посвящена  исследованию рождения предельного цикла из положения равновесия с использованием метода Андронова-Хопфа для популяционной модели Мэя с неограниченным и ограниченным запаздыванием.
 В работе изучается популяционная модель Мэя. Для нахождения предельных циклов используется метод Хопфа. Рассмотрен случай неограниченного последействия. Исследовалась устойчивость периодических колебаний.


\cleardoublepage                     
\pagenumbering{gobble}         

\tableofcontents

\cleardoublepage                   
\pagenumbering{arabic}         

\setcounter{page}{4}  

\section*{ВВЕДЕНИЕ}
\addcontentsline{toc}{section}{ВВЕДЕНИЕ}

Введение
Воздействие человека на природу неуклонно растет. Избежать его принципиально невозможно, поэтому необходимо искать возможности минимизировать вредные последствия на природные системы. Поэтому актуальны и необходимы математические модели систем и математические методы их анализа \cite{baz}.\\
Первую математическую модель для описания динамики численности вида предложил в 1798 г. Т.Мальтус. Согласно этой модели при благоприятных условиях любой вид увеличивает свою численность по экспоненциальному закону, который удовлетворяет дифференциальному уравнению 
\begin{equation*}
    \frac{d N(t)}{d t}=r N(t),
\end{equation*}
где параметр  r   в последующем получил название мальтузианского коэффициента линейного роста \cite{Dolgiy}. Модель Мальтуса хорошо подтверждается экспериментальным материалом, если численность популяции невелика. \\
В 1948г. Г.Хатчинсоном была предложена модель, описываемая дифференциальным уравнением с запаздыванием
\begin{equation*}
    \frac{d N(t)}{d t}=r \left(1 - \frac{N(t-h)}{K} \right) N(t),
\end{equation*}
где h параметр, характеризующий средний размер репродуктивного возраста вида, r – коэффициент размножения, K – емкость среды обитания. Идея модели заключается в том, что скорость изменения популяции определяется численностью популяции в некоторый предшествовавший момент времени \cite{Dolgiy}. \\
Популяционная модель Лотки-Вольтерра существования двух видов, первый из которых является травоядным (жертва), а второй плотоядным(хищник), описывается системой обыкновенных дифференциальных уравнений 
\begin{equation*}
    \frac{d N_1 (t)}{d t}=N_1 (\epsilon_1 - \gamma_1 N_2)
\end{equation*}
\begin{equation*}
    \frac{d N_2 (t)}{d t}= -N_2 (\epsilon_2 - \gamma_2 N_2),
\end{equation*}
где $N_1(t)$,$N_2(t)$ - численности жертвы и хищника, соответственно в момент времени t. Параметры системы $\epsilon_1$,$\epsilon_2$,$\gamma_1$,$\gamma_2$ - положительные параметры \cite{has}
Популяционная модель Лотки-Вольтерра  имеет семейство периодических решений, периоды которых зависят от начальных условий. Наблюдения показали, что циклические процессы в системах “хищник-жертва” описываются предельными циклами, периодическими решениями с изолированными траекториями. Моделированию таких популяционных систем уделяется большое внимание в научной литературе \cite{has}.  Особое внимание уделяется вопросу устойчивости этих систем\cite{svir} . В настоящей работе изучается популяционная модель Мэя. Для нахождения предельных циклов используется метод Хопфа. Рассмотрен случай неограниченного последействия. Исследовалась устойчивость периодических колебаний.


\newpage
\section{Бифуркация Хопфа для модели Мэя с неограниченным запаздыванием}
\addcontentsline{toc}{section}{Бифуркация Хопфа для модели Мэя с неограниченным запаздыванием}
\subsection{Описание популяционной модели Мэя}
В популяционной модели <<хищник-жертва>> Мэя присутствует распределенное запаздывание  \cite{has}.
\begin{equation}
    \frac{d N(t)}{d t}=r N(t) \left( \left(1+ \int_{-\infty}^{0}Q(-s) N(t+s) d s \right)-\alpha P(t) \right), 
    \end{equation}
  \begin{equation*}
    \frac{d P(t)}{d t}=\left( -b + \beta N(t)\right) P(t),
\end{equation*}
где параметры $\alpha$,$\beta$,r,b - положительны; Q - заданная функция, описывающая вызревание пищи (салата); N(t) - описывает численность популяции травоядных (кроликов); P(t)- хищников (лисиц).\\
Рассмотрим функцию Q(t) частного вида:
\begin{equation*}
    Q(t)=\frac{t}{T^2}\exp{\frac{-t}{T}}, t>0 ,
\end{equation*}
где T - среднее время восстановления растительности после выедания. Тогда первое уравнение системы (1) будет иметь вид:
\begin{equation*}
     \frac{d N(t)}{d t}= N(t) \left(r \left(1+ \int_{-\infty}^{0}\frac{s}{T^2}\exp{\frac{-s}{T}} N(t+s) d s \right)-\alpha P(t) \right)
\end{equation*} \\
Используя масштабирование в системе (1) t$\to$Tt,s$\to$Ts,N(Tt)$\to$N(t),P(Tt)$\to$P(t) получаем систему:
\begin{equation}
     \frac{d N(t)}{d t}=T N(t) \left(r \left(1+ \int_{-\infty}^{0}s\exp{s} N(t+s) d s \right)-\alpha P(t) \right)
\end{equation}
\begin{equation*}
     \frac{d P(t)}{d t}=T\left( -b + \beta N(t)\right) P(t),
\end{equation*}
\newpage
\subsection{Бифуркация Хопфа}
\ Мы изучим бифуркацию рождения периодического решения из положения равновесия для масштабированной популяционной модели Мэя (2).\\
\ Находим положение равновесия системы $N(t)=N_0=const >0$ ; $P(t)=P_0=const>0$
\begin{equation*}
    0=T N_0 \left(r \left(1+ \int_{-\infty}^{0}s\exp{s} N_0 d s \right)-\alpha P_0 \right)
\end{equation*}
\begin{equation*}
    0=T\left( -b + \beta N_0\right) P_0.
\end{equation*}\\
Из этой системы находим одно положение равновесия 
\begin{equation}
    N_0= \frac{b}{\beta}; P_0=\frac{r}{\alpha}(1-\frac{b}{\beta})
\end{equation}
 Используя замену $N(t)=N(t)+N_0 ; P(t)=P(t)+P_0$ получаем  систему уравнений возмущенного движения  в линейном приближении:
 \begin{equation}
    \frac{d N(t)}{d t}=T N_0 \left(r  \int_{-\infty}^{0}s e^{s} N(t+s) d s -\alpha P(t) \right)
 \end{equation}
 \begin{equation*}
      \frac{d P(t)}{d t}=T \beta N(t) P_0
 \end{equation*}\\
 Характеристическое уравнение такой системы имеет вид:
 \begin{equation*}
     \lambda^2 + r T N_0 \frac{\lambda}{(1+\lambda)^2}+T^2 b r (1-\frac{b}{\beta})=0
 \end{equation*}
 Преобразуя полученное выражение, получим:
 \begin{equation}
     \lambda^4 +2 \lambda^3+\lambda^2(1+T^2 b r (1-\frac{b}{\beta}))+\lambda(2 T^2 b r (1-\frac{b}{\beta})+T r \frac{b}{\beta}) + T^2 b r (1-\frac{b}{\beta})=0
 \end{equation}
 Для существования периодических решений должны выполняться 2 условия Хопфа \cite{has}:
 \begin{enumerate}
 \item   Существуют $\lambda_{1,2}$=$\pm$ i $\nu$\\
 \item $ Re\lambda_1 \neq 0$, где $\lambda$ =$\pm$ i $\nu$ + $\lambda_1 \mu$ + O ($\mu^2)$\\
 \end{enumerate}
 Оба условия Хопфа выполняются при наложении на параметры условия связи:
 \begin{equation*}
     \beta=\frac{1}{T(1-k)} \left(\frac{1}{T r k} - \frac{1}{2}\right),
\end{equation*}
 где $k=\frac{b}{\beta}$, и при этом $\nu=1$. 
 Дальше будем искать периодическое решение через независимые параметры r,T,k. Возмущаем параметр T=T+$\mu$.
Производим замену функции P(t) для исключения параметра $\alpha$ из уравнений системы : $ P(t)=\frac{P(t)}{\alpha}$. На основании этих преобразований получаем систему возмущенного движения.
\newpage
\subsection{Вычисление периодического движения}
 Имеем систему возмущенного движения:
  \begin{equation}
      \frac{d N(t)}{d t}= (T+\mu)(N(t)+k)\left( r\int_{-\infty}^{0} s e^s  N(t+s) d s - P(t) \right)
  \end{equation}
\begin{equation*}
      \frac{d P(t)}{d t}= (T+\mu)N(t)( r(1-k) + P(t) )\left( \frac{1}{T r k} - \frac{1}{2}\right)\frac{1}{T(1-k)}
  \end{equation*}
  Требуется найти периодическое решение системы (6) для периода $\omega=2 \pi (1+\alpha)$,
   \begin{equation*}
       \alpha=\alpha (\mu); a(0)=0.
       \end{equation*}
Выполняется замена  переменных:
\begin{equation*}
    t=(1+\alpha)t; 
\end{equation*}
С учетом этой замены, система примет вид:
\begin{equation} 
      \frac{d N(t)}{d t}=(1+\alpha) (T+\mu)(N(t)+k)\left( (1+\alpha)^2 r\int_{-\infty}^{0} s e^{s (1+\alpha)}  N(t+s) d s - P(t) \right)
\end{equation}
\begin{equation*}
      \frac{d P(t)}{d t}=(1+\alpha) (T+\mu)N(t)( r(1-k) + P(t) )\left( \frac{1}{T r k} - \frac{1}{2}\right)\frac{1}{T(1-k)}
  \end{equation*}
  Введем обозначения: $N(t)=y_1 (t)$, $P(t)=y_2 (t)$ $y(t)=(y_1 (t); y_2(t))^T$. 2$\pi$ -периодические решения $y(t,\gamma)$ системы (7), а также параметры этой системы $\mu(\gamma)$ и $\alpha(\gamma)$ будем искать в форме асимптотических рядов по вспомогательному параметру $\gamma$ \cite{has}, т.е.
  \begin{equation}
      y(t,\gamma)=\sum_{k=1}^n \gamma^k y^k (t)= \gamma y(t)+\gamma^2 y^2(t)+\gamma^3 y^3(t)+\ldots
 \label{math/2}
  \end{equation}
   \begin{equation} \label{mu}
       \mu(\gamma)=\sum_{k=1}^n \gamma^k \mu_k = \gamma \mu_1+\gamma^2 \mu_2+\gamma^3 \mu_3+\ldots
  \end{equation}
  \begin{equation}
       \alpha(\gamma)=\sum_{k=1}^n \gamma^k \alpha_k = \gamma \alpha_1+\gamma^2 \alpha_2+\gamma^3 \alpha_3+\ldots
  \end{equation}
  Тогда исходная система запишется в виде:
  \begin{multline}
      \label{system1}
       \frac{d y_1(t)}{d t}=(1+\alpha_1 \gamma+\ldots) (T+\mu_1 \gamma+\ldots)(y_1^1(t) \gamma+\ldots+k)( (1+\\+\alpha_1 \gamma++\ldots)^2 r  \int_{-\infty}^{0} s e^{s (1+\alpha_1 \gamma + \ldots)} ( y_1^1(t+s)\gamma +\ldots)d s -\\- (y_2^1(t) \gamma+\ldots ))
  \end{multline}
  \begin{multline*}
     \frac{d y_2(t))}{d t}=(1+\alpha_1 \gamma+\ldots) (T+\mu_1 \gamma+\ldots)(y_1^1(t) \gamma + \ldots)( r(1-k) +\\+( y_2^1(t)\gamma+\ldots) )\left( \frac{1}{T r k} - \frac{1}{2}\right)\frac{1}{T(1-k)}
  \end{multline*}
  Далее строим первое приближение:
  \begin{equation}\label{s1}
      \frac{d y^1_1(t)}{d t}=T k (r \int_{-\infty}^{0} s e^{s}y^1_1(t+s)d s -  y^1_2 (t) ) 
  \end{equation}
\begin{equation*}
   \frac{d y^1_2(t)}{d t}= r(\frac{1}{T r k} -\frac{1}{2})y^1_1(t) ,
\end{equation*}
 Используем решение Эйлера для получения решения системы первого приближения:
 \begin{equation*}
    y_1^1=C_1^1 e^{\lambda t}
 \end{equation*}
 \begin{equation*}
    y_2^1=C_2^1 e^{\lambda t}
 \end{equation*}
 Характеристическое уравнение данной системы имеет 2 мнимых корня, которым отвечают собственные решения 
 \begin{equation*}
    y_1^1=C_1^1 e^{i t}+C_1^{-1} e^{-i t}
 \end{equation*}
 \begin{equation*}
    y_2^1=C_2^1 e^{+i t}+C_2^{-1} e^{-i t}
 \end{equation*}
 Постоянные $C_1^1$, $C_2^1$, определяются из системы алгебраических уравнений:
 \begin{equation*}
     i C_1^1=T k r C_1^1 \frac{i}{2}- T k C_2^1
 \end{equation*}
 \begin{equation*}
     i C_2^1=r \left(  \frac{1}{T r k} - \frac{1}{2}\right) C_1^1
 \end{equation*}
 Решение системы первого приближения будет выглядеть следующим образом:
 \begin{equation*}
    y_1^1= e^{i t}+ e^{-i t}
 \end{equation*}
 \begin{equation*}
    y_1^2=-i r \left(  \frac{1}{T r k} - \frac{1}{2}\right) e^{+i t}+ i r \left(  \frac{1}{T r k} - \frac{1}{2}\right) e^{-i t}
 \end{equation*}
 Далее рассматриваем следующее приближение:
\begin{equation}\label{s2}
       \frac{d y^2_1(t)}{d t}=T k (r \int_{-\infty}^{0} s e^{s}y^2_1(t+s)d s -  y^2_2 (t) ) + f^2_1 (t)
   \end{equation}
   \begin{equation*}
   \frac{d y^2_2(t)}{d t}= r(\frac{1}{T r k} -\frac{1}{2})y^2_1(t)+f^2_2 (t) ,
\end{equation*}
где $ f^2_1 $ и $f^2_2$ определяются формулами:
\begin{multline}
    f^2_1 (t)=\alpha_1 T k \left( r \int_{-\infty}^{0} s e^{s} y^1_1 (t+s) d s - y^1_2 (t) \right)+\mu_1 k \left( r \int_{-\infty}^{0} s e^{s} y^1_1 (t+s) d s - y^1_2 (t) \right)\\ +T y^1_1 (t) \left( r \int_{-\infty}^{0} s e^{s} y^1_1 (t+s) d s - y^1_2 (t) \right) \\+T k r \left(  2  \alpha_1 \int_{-\infty}^{0} s e^{s} y^1_1 (t+s)d s +  \alpha_1 \int_{-\infty}^{0}s^2 e^{s} y^1_1(t+s) d s     \right) ,
    \end{multline}
    \begin{equation*}
    f^2_2(t)=\frac{1}{T (1-k)}\left( \frac{1}{T k r} - \frac{1}{2} \right)\left[ \alpha_1 T y^1_1(t)r(1-k)+\mu_1 y^1_1 (t)r(1-k) + T y^1_1(t) y^1_2(t)       \right].
    \end{equation*}
Полученные выражения можно разложить по резонансным гармоникам:
 \begin{equation}
      f^2_1(t)=d^1_1 e^{i t}+d^2_1 e^{2 i t} +d^{-1}_1e^{-i t}+d^{-2}_1 e^{-2 i t}
    \end{equation}
    \begin{equation*}
      f^2_2(t)=d^1_2 e^{i t}+d^2_2 e^{2 i t} +d^{-1}_2e^{-i t}+d^{-2}_2 e^{-2 i t},
    \end{equation*}
где коэффициенты разложения $ d^1_1 $, $ d^1_2 $, $ d^2_1$ и $d^2_2$ определяются формулами :
\begin{equation*}
    d^1_1=\alpha_1 T k r \left ( -0.5+i(\frac{1}{2}+\frac{1}{T k r}) \right)+\frac{i}{T} \mu_1,
\end{equation*}
\begin{equation*}
    d^{-1}_1=\alpha_1 T k r \left ( -0.5-i(\frac{1}{2}+\frac{1}{T k r}) \right)-\frac{i}{T} \mu_1,
\end{equation*}
\begin{equation*}
    d^2_1=\frac{i}{k}=- d^{-2}_{1},
\end{equation*}
\begin{equation*}
    d^1_2=\alpha_1  r \left ( \frac{1}{T k r}-\frac{1}{2} \right)+\frac{r}{T} \mu_1 \left ( \frac{1}{T k r}-\frac{1}{2} \right)=d^{-1}_2,
\end{equation*}
\begin{equation*}
    d^2_2=\frac{-i r}{1-k} \left ( \frac{1}{T k r}-\frac{1}{2} \right)^2 =-d^{-2}_2
\end{equation*}
Условием существования 2$\pi$ -периодического решения является уравнение отбора:
\begin{equation*}
    \psi^* d^1 =0,
\end{equation*}
где $\psi$ - вектор решение сопряженной системы :
 \begin{equation*}
     i \psi_1^1=T k r \psi_1^1 \frac{i}{2}- T k \psi_2^1
 \end{equation*}
 \begin{equation*}
     i \psi_2^1=r \left(  \frac{1}{T r k} - \frac{1}{2}\right) \psi_1^1 ,
 \end{equation*}
 Откуда получаем
 \begin{equation*}
     \psi_1^1 =1
 \end{equation*}
 \begin{equation*}
     \psi_2^1=-i r \left(  \frac{1}{T r k} - \frac{1}{2}\right)
 \end{equation*}
Условие существование 2$\pi$-периодического решения принимает вид:
\begin{equation}
    \psi_1^{1*} d^1_1+\psi_2^{1*} d^1_2 =0
\end{equation}
\begin{multline*}
   \alpha_1 T k r \left ( -0.5+i(\frac{1}{2}+\frac{1}{T k r}) \right)+\frac{i}{T} \mu_1+\\+i r \left(  \frac{1}{T r k} - \frac{1}{2}\right) \left(\alpha_1  r \left ( \frac{1}{T k r}-\frac{1}{2} \right)++\frac{r}{T} \mu_1 \left ( \frac{1}{T k r}-\frac{1}{2} \right)\right) =0
\end{multline*}
Так как все параметры являются вещественными числами, то уравнение разделяется на систему алгебраических уравнений, путем приравнивания мнимой и действительной частей левого выражения к 0:
\begin{equation*}
    \alpha_1 (-0.5) T k r=0,
\end{equation*}
\begin{equation*}
    \frac{\mu_1}{T}\left( 1+r^2 \left(\frac{1}{T r k}-\frac{1}{2} \right)^2\right)+\alpha_1 \left(T k r \left(\frac{1}{2}+\frac{1}{ T k r}\right)+r^2 \left(\frac{1}{2}-\frac{1}{T k r}\right)^2 \right)=0.
\end{equation*}
Откуда получаем, что $\alpha_1$, $\mu_1$ равны 0. Необходимо рассматривать следующее приближение. Поиск периодических решений заканчивается на k-ом шаге, когда $\mu_k$ $\neq$ 0.
Для продолжения необходимо найти второй член разложения решения:
\begin{equation*}
     y^2_1= C_1^2 e^{2 i t}+ C_1^{-2} e^{-2 i t}
  \end{equation*}
  \begin{equation*}
     y^2_2= C_2^2 e^{2 i t}+ C_2^{-2} e^{-2 i t}
  \end{equation*}
  Постоянные $C_1^2$,$C_2^2$ определяются из системы алгебраических уравнений:
  \begin{equation*}
     2i C_1^2=T k r C_1^2( \frac{4i}{25}+\frac{3}{25})- T k C_2^2 + \frac{i}{k}
 \end{equation*}
 \begin{equation*}
     2i C_2^2=r \left(  \frac{1}{T r k} - \frac{1}{2}\right) C_1^2-\frac{i r}{1-k} \left(\frac{1}{T r k} - \frac{1}{2} \right)^2
 \end{equation*}
 Откуда получаем следующие результаты:
\begin{equation*}
    C^2_1 =\frac{1}{(1-k)(144 T^2 k^2 r^2 + (150+9 T k r)^2)}[a_1 + i b_1]
\end{equation*}
\begin{equation*}
    C^2_2 =\left(\frac{1}{2}-\frac{1}{T r k}\right)\frac{1}{2(1-k)T k (144 T^2 k^2 r^2 + (150+9 T k r)^2)}\left[a_2+ i b_2 \right]
\end{equation*}
где $a_1$,$b_1$,$a_2$,$b_2$ имеют следующий вид
\begin{equation*}
    a_1=-150 T^2 k^2 r^2 -300 T k r +900 T r + \frac{15000}{k} -15600
\end{equation*}
\begin{equation*}
    b_1=7050-\frac{7500}{T k r}-1200 T r -225 T k r -112.5 T^2 k^2 r^2
\end{equation*}
\begin{equation*}
    a_2=30000-900T^2 k^2 r^2-15600 k r T +1200 k r^2 T^2
\end{equation*}
\begin{equation*}
    b_2= -150T^3 k^3 r^3-300 T^2 k^2 r^2 +900 T^2 k r^2  - 15600 T k r+15000 T r
\end{equation*}

Далее строим третье приближение
\begin{equation}\label{s3}
       \frac{d y^3_1(t)}{d t}=T k \left(r \int_{-\infty}^{0} s e^{s}y^3_1(t+s)d s -  y^3_2 (t) \right) + f^3_1 (t)
   \end{equation}
   \begin{equation*}
   \frac{d y^3_2(t)}{d t}= r\left(\frac{1}{T r k} -\frac{1}{2}\right)y^3_1(t)+f^3_2 (t) ,
\end{equation*}
где $ f^3_1 $ и $f^3_2$ определяются формулами:
\begin{multline}
    f^3_1 (t)=\alpha_2 T k \left( r \int_{-\infty}^{0} s e^{s} y^1_1 (t+s) d s - y^1_2 (t) \right)+\mu_2 k \left( r \int_{-\infty}^{0} s e^{s} y^1_1 (t+s) d s - y^1_2 (t) \right)\\ +T y^1_1 (t) \left( r \int_{-\infty}^{0} s e^{s} y^2_1 (t+s) d s - y^2_2 (t) \right) \\+T k r \left(  2  \alpha_2 \int_{-\infty}^{0} s e^{s} y^1_1 (t+s)d s +  \alpha_2 \int_{-\infty}^{0}s^2 e^{s} y^1_1(t+s) d s     \right) \\+T y^2_1 \left( r \int_{-\infty}^{0}s e^s y^1_1(t+s) d s - y^1_2(t)\right),
    \end{multline}
    \begin{equation*}
    f^3_2(t)=\frac{1}{T (1-k)}\left( \frac{1}{T k r} - \frac{1}{2} \right)\left[ \alpha_1 T y^1_1(t)r(1-k)+\mu_1 y^1_1 (t)r(1-k) + T( y^1_1(t) y^2_2(t)     + y^2_1 y^1_2  )\right].
    \end{equation*}
Полученные выражения можно разложить по резонансным гармоникам:
 \begin{equation}
      f^3_1(t)=d^1_1 e^{i t}+d^3_1 e^{3 i t} +d^{-1}_1e^{-i t}+d^{-3}_1 e^{-3 i t}
    \end{equation}
    \begin{equation*}
      f^3_2(t)=d^1_2 e^{i t}+d^3_2 e^{3 i t} +d^{-1}_2e^{-i t}+d^{-3}_2 e^{-3 i t},
    \end{equation*}
где коэффициенты разложения $ d^1_1 $ и $ d^1_2 $, определяются формулами :
\begin{equation*}
    d^1_1=\alpha_2 T k r \left ( -0.5+i(\frac{1}{2}+\frac{1}{T k r}) \right)+\frac{i}{T} \mu_2+d_1,
\end{equation*}
\begin{equation*}
    d^1_2=\alpha_2  r \left ( \frac{1}{T k r}-\frac{1}{2} \right)+\frac{r}{T} \mu_2 \left ( \frac{1}{T k r}-\frac{1}{2} \right)+d_2,
\end{equation*}
где 
\begin{equation*}
     d_1=\frac{1}{k(1-k)T k r (144 T^2 k^2 r^2 + (150+9 T k r)^2)}[ P_1 + i Q_1]
\end{equation*}
\begin{equation*}
    d_2=\frac{(T k r -2)^2}{8(1-k)^2 T^3 k^3 r^2 (144 T^2 k^2 r^2 + (150+9 T k r)^2)}[ P_2 + i Q_2]
\end{equation*}
а $P_1$, $Q_1$, $P_2$, $A_2$ определяются формулами
\begin{equation*}
    P_1 = 7500-7050 T k r +1200 T^2 r^2 k + 225 T^2 r^2 k^2 + 112.5 T^3 r^3 k^3
\end{equation*}
\begin{equation*}
   Q_1 = -7500 r T +6900 k r T-1800 k r^2 T^2 + 2400 k^2 r^2 T^2-225 k^2 r^3 T^3+75 k^3 r^3 T^3
\end{equation*}
\begin{equation*}
   P_2= 225 k^3 r^3 T^3+1350k^2 r^2 T^2 +1200 k r^2 T^2+1500 k r T-15000
\end{equation*}
\begin{equation*}
    Q_2=-150T^3 k^3 r^3-300 T^2 k^2 r^2 +900 T^2 k r^2  - 15600 T k r+15000 T r
\end{equation*}

Условием существования 2$\pi$ -периодического решения является уравнение отбора:
\begin{equation*}
    \psi^* d^1 =0,
\end{equation*}
где $\psi$ - вектор решение сопряженной системы :
 \begin{equation*}
     i \psi_1^1=T k r \psi_1^1 \frac{i}{2}- T k \psi_2^1
 \end{equation*}
 \begin{equation*}
     i \psi_2^1=r \left(  \frac{1}{T r k} - \frac{1}{2}\right) \psi_1^1 ,
 \end{equation*}
 Откуда получаем
 \begin{equation*}
     \psi_1^1 =1
 \end{equation*}
 \begin{equation*}
     \psi_2^1=-i r \left(  \frac{1}{T r k} - \frac{1}{2}\right)
 \end{equation*}
Условие существование 2$\pi$-периодического решения принимает вид:
\begin{equation}
    \psi_1^{1*} d^1_1+\psi_2^{1*} d^1_2 =0
\end{equation}
\begin{multline*}
   \alpha_2 T k r \left ( -0.5+i\left(\frac{1}{2}+\frac{1}{T k r}\right) \right)+\frac{i}{T} \mu_2+d_1\\+i r \left(  \frac{1}{T r k} - \frac{1}{2}\right) \left(\alpha_2  r \left ( \frac{1}{T k r}-\frac{1}{2} \right)+\frac{r}{T} \mu_2 \left ( \frac{1}{T k r}-\frac{1}{2} \right)+d_2\right) =0
\end{multline*}
\begin{multline*}
  \frac{-1}{2} \alpha_2 T k r + Re d_1 + \left(\frac{1}{2}-\frac{1}{T k r}\right) r Im d_2+ \\+i(\frac{\mu_2}{T}+\alpha_2 T k r \left(\frac{1}{2}-\frac{1}{T k r}\right) + Im d_1-\\- \left(\frac{1}{2}-\frac{1}{T k r}\right)^2 r^2 \alpha_2-\left(\frac{1}{2}-\frac{1}{T k r}\right)^2 r^2 \frac{\mu_2}{T}-
 \left(\frac{1}{2}-\frac{1}{T k r}\right)r Re d_2 )=0
\end{multline*}
Так как все параметры являются вещественными числами, то уравнение разделяется на систему алгебраических уравнений, путем приравнивания мнимой и действительной частей левого выражения к 0:
\begin{equation*}
    \frac{-1}{2} \alpha_2 T k r + Re d_1 + \left(\frac{1}{2}-\frac{1}{T k r}\right) r Im d_2=0,
\end{equation*}
\begin{multline*}
 \frac{\mu_2}{T}+\alpha_2 T k r \left(\frac{1}{2}-\frac{1}{T k r}\right) + Im d_1-\\- \left(\frac{1}{2}-\frac{1}{T k r}\right)^2 r^2 \alpha_2-\left(\frac{1}{2}-\frac{1}{T k r}\right)^2 r^2 \frac{\mu_2}{T}-
 \left(\frac{1}{2}-\frac{1}{T k r}\right)r Re d_2 =0
\end{multline*} 
Откуда получаем:
\begin{equation*}
    \alpha_2 =\frac{2}{T k r}\left( Re d_1 + \left(\frac{1}{2}-\frac{1}{T k r}\right) r Im d_2\right),
\end{equation*}
\begin{equation*}
 \alpha_2 =\frac{1}{16 (1-k)^2 T^3 k^3 r(144 T^2 k^2 r^2 + (150+9 T k r)^2)} \alpha^*,
\end{equation*}
где $\alpha^*$:
\begin{multline*}
    \alpha^*=124800-\frac{120000}{k}+172800 r T -184800 T r k+120000 T^2 k - 120000 T^2 k^2 - 79200 T^2 r^2 k +\\+91200 T^2 r^2 k^2 -112800 T^3 k^2 r +112800 T^3 r k^3 + 9600 T^3 r^3 k^2 -15600 T^3 r^3 k^3 + 19200 T^4 r^2 k^2 -\\-15600 T^4 r^2 k^3 -3600 T^4 r^2 k^4 +900 T^4 r^4 k^3 +600 T^4 r^4 k^4 + 1800 T^5 r^3 k^4-\\ -1800 T^5 r^3 k^5 -150 T^5 r^5 k^5
\end{multline*}
 Выражая из 2 уравнения системы $\mu_2$ и подставляя полученные ранее значения всех слагаемых, получаем следующий результат:
 \begin{equation}
 \mu_2 =\frac{\mu_2^*}{8(1-k)^2 T^3 k^3 r^2 (144 T^2 k^4 r^2 + (150+9 T k r)^2)(4 T^2 k^2 + (T r k - 2)^2)} ,
\end{equation}
где $\mu^*$:
\begin{multline*}
    \mu^*=750. k^8 r^6 T^8+6000. k^8 r^4 T^8+150 k^7 r^7 T^7+1200. k^7 r^5 T^7-13200. k^7 r^4 T^8+\\+91200. k^7 r^3 T^7-1200 k^6 r^6 T^6-1200. k^6 r^5 T^7+7200 k^6 r^4 T^8+17400. k^6 r^4 T^6-110400. k^6 r^3 T^7+\\+9600. k^6 r^2 T^6-900 k^5 r^6 T^6+18600 k^5 r^5 T^5-14400. k^5 r^4 T^6+19200. k^5 r^3 T^7-259200. k^5 r^3 T^5-\\-172800. k^5 r^2 T^6-211200. k^5 r T^5-6000 k^4 r^5 T^5-19200. k^4 r^4 T^6-156000 k^4 r^4 T^4+\\+192000. k^4 r^3 T^5+163200. k^4 r^2 T^6+784800. k^4 r^2 T^4+211200. k^4 r T^5+480000. k^4 T^4+\\+114000 k^3 r^4 T^4+76800. k^3 r^3 T^5+612000 k^3 r^3 T^3-556800. k^3 r^2 T^4-825600. k^3 r T^3-\\-480000. k^3 T^4-528000 k^2 r^3 T^3-76800. k^2 r^2 T^4-1228800 k^2 r^2 T^2+480000. k^2 r T^3+\\+220800. k^2 T^2+1128000 k r^2 T^2+1238400 k r T+\frac{480000}{k}-1171200 r T+499200.
\end{multline*}
Для оценки значений параметра $\mu_2$ в зависимости от значений  параметров системы, введем параметр $T_1$=$T r k$, на который, исходя из постановки задачи, накладывается ограничение:
\begin{equation*}
    0<T_1 <2
\end{equation*}
Также из постановки задачи имеем ограничение:
\begin{equation*}
    0<k<1 
\end{equation*}
Построим области, где $\mu_2$ отрицательно. Мы считаем, что $\mu$ - возмущение параметра T, положительно. Таким образом мы пересекаем границу устойчивости положения равновесия и попадаем в область неустойчивости, где возможны периодические движения. Тогда исходя из разложения (\ref{mu}) параметра $\mu$ по вспомогательному параметру $\gamma$ следует, что $\mu_2$ должно быть положительно для существования периодического решения.
\begin{figure}
    \centering
    \includegraphics{mu2area}
    \caption{Выделенные области соответствуют отрицательным значения параметра $\mu_2$}
    \label{fig:mu2}
\end{figure}
Как видно из графика значений (Рис.\ref{fig:mu2}) существуют области, где $\mu_2$ - отрицательна. Но этим областям отвечают граничные значения параметров $T_1$,$r$,$k$. Эти области могут соответствовать периодическим решениям, которые появляются внутри области устойчивости, когда мы не переходим границ области устойчивости. Для исключения данных областей из рассмотрения достаточно наложить условия на ограниченность параметров:
\begin{align*}
    0.1<k<0.9\\
    1.2 < r\\
    0.05 < T1 < 1.9995
\end{align*}
\newpage
\section{Устойчивость периодического решения}
Перепишем систему (\ref{system1}), полагая, что $y(t)=y(t+\gamma)+z(t)$
 \begin{multline}
      \label{system1u}
       \frac{d (y_1(t)+z_1 (t))}{d t}=(1+\alpha) (T+\mu )(y_1(t)+z_1(t) +k)( (1+\alpha)^2 r  \int_{-\infty}^{0} s e^{s (1+\alpha)} ( y_1(t+s)+\\+z_1(t+s))d s - (y_2(t) +z_2(t) ))
  \end{multline}
  \begin{multline*}
     \frac{d (y_2(t)+z_2(t))}{d t}=(1+\alpha) (T+\mu)(y_1(t)+z_1(t))( r(1-k) +( y_2(t)+\\+z_2(t)) )\left( \frac{1}{T r k} - \frac{1}{2}\right)\frac{1}{T(1-k)}
  \end{multline*}
  Отсюда составляем систему уравнений возмущенного движения:
  \begin{multline}
      \label{system1z}
       \frac{d z_1 (t)}{d t}=(1+\alpha) (T+\mu )(y_1(t) +k)\left( (1+\alpha)^2 r  \int_{-\infty}^{0} s e^{s}(1+\alpha s)  z_1(t+s)d s -z_2(t) \right)+\\+(1+\alpha) (T+\mu )z_1(t)\left( (1+\alpha)^2 r  \int_{-\infty}^{0} s e^{s}(1+\alpha s) ( y_1(t+s)+z_1(t+s))d s -(y_2(t)+z_2(t)) \right)
  \end{multline}
  \begin{multline*}
     \frac{d z_2(t)}{d t}=(1+\alpha) (T+\mu)(z_1(t)( r(1-k) + y_2(t)+z_2(t))+y_1(t)z_2(t) )\left( \frac{1}{T r k} - \frac{1}{2}\right)\frac{1}{T(1-k)}
  \end{multline*}
  Система уравнений линейного приближения имеет вид:
  \begin{multline}
      \label{system1zl}
       \frac{d z_1 (t)}{d t}=(1+\alpha) (T+\mu )((y_1(t) +k)\left( (1+\alpha)^2 r  \int_{-\infty}^{0} s e^{s}(1+\alpha s)  z_1(t+s)d s -z_2(t) \right)+\\+z_1(t)\left( (1+\alpha)^2 r  \int_{-\infty}^{0} s e^{s}(1+\alpha s)  y_1(t+s)d s -y_2(t) \right))
  \end{multline}
  \begin{multline*}
     \frac{d z_2(t)}{d t}=(1+\alpha) (T+\mu)(z_1(t)( r(1-k) + y_2(t))+y_1(t)z_2(t) )\left( \frac{1}{T r k} - \frac{1}{2}\right)\frac{1}{T(1-k)}
  \end{multline*}
  В случае $\gamma=0$ система принимает вид:
  \begin{equation}
      \frac{d z_1(t)}{d t}=T k (r \int_{-\infty}^{0} s e^{s}z_1(t+s)d s -  z_2 (t) ) 
  \end{equation}
\begin{equation*}
   \frac{d z_2(t)}{d t}= r(\frac{1}{T r k} -\frac{1}{2})z_1(t).
\end{equation*}
Для такой системы (\ref{s1}) мы уже показывали существование двух чисто мнимых корней $\pm i$ характеристического уравнения. Им отвечает кратный характеристический показатель $\lambda_0 (0) =0$. Согласно теореме Андронова-Витте \cite{has}, устойчивость периодического решения определяется ненулевым характеристическим показателем $\lambda (\gamma)$, для которого имеет место асимптотическое разложение:
\begin{equation}
    \lambda (\gamma)=\lambda_1 \gamma + \lambda_2 \gamma^2 + O(\gamma^3)
\end{equation}
Этому асимптотическому разложению отвечает решение Флоке:
\begin{equation*}
 z(t, \gamma)=u(t, \gamma)e^{\lambda (\gamma) t},
\end{equation*}
где $u(t+2 \pi, \gamma)=u(t, \gamma)$.\\
С помощью него найдем коэффициенты асимптотического разложения для определения областей устойчивости периодического движения.Нахождение решения Флоке сводится к нахождению 2$\pi$-периодического решения системы уравнений
\begin{multline}
      \label{system1ul}
       \frac{d u_1 (t)}{d t}=(1+\alpha) (T+\mu )((y_1(t) +k)( (1+\alpha)^2 r  \int_{-\infty}^{0} s e^{s}(1+(\alpha + \lambda (\gamma))s)  u_1(t+s)d s -\\-u_2(t))+u_1(t)\left( (1+\alpha)^2 r  \int_{-\infty}^{0} s e^{s}(1+\alpha s)  y_1(t+s)d s -y_2(t) \right))-u_1 (t) \lambda (\gamma)
  \end{multline}
  \begin{multline*}
     \frac{d u_2(t)}{d t}=(1+\alpha) (T+\mu)(u_1(t)( r(1-k) + y_2(t))+y_1(t)u_2(t) )\left( \frac{1}{T r k} - \frac{1}{2}\right)\frac{1}{T(1-k)}-\\-u_2 (t) \lambda (\gamma)
  \end{multline*}
Периодическое решение системы (\ref{system1ul}) ищем с помощью асимптотического разложения:
\begin{equation*}
    u(t, \gamma)= u^0 (t) + u^1 (t)\gamma + u^2(t) \gamma^2 + O(\gamma^3)
\end{equation*}
Первое приближение нашей системы имеет вид:
 \begin{equation} \label{su0}
      \frac{d u_1^0(t)}{d t}=T k (r \int_{-\infty}^{0} s e^{s}u_1^0(t+s)d s -  u_2^0 (t) ) 
  \end{equation}
\begin{equation*}
   \frac{d u_2^0(t)}{d t}= r(\frac{1}{T r k} -\frac{1}{2})u_1^0(t).
\end{equation*}
 Используем решение Эйлера для получения решения системы первого приближения:
  \begin{equation*}
    u_1^0=C_1^1 e^{i t}+C_1^{-1} e^{-i t}
 \end{equation*}
 \begin{equation*}
    u_2^0=C_2^1 e^{+i t}+C_2^{-1} e^{-i t}
 \end{equation*}
 Постоянные $C_1^1$, $C_2^1$, определяются из системы алгебраических уравнений:
 \begin{equation*}
     i C_1^1=T k r C_1^1 \frac{i}{2}- T k C_2^1
 \end{equation*}
 \begin{equation*}
     i C_2^1=r \left(  \frac{1}{T r k} - \frac{1}{2}\right) C_1^1
 \end{equation*}
 Решение системы первого приближения будет выглядеть следующим образом:
 \begin{equation*}
    u_1^0= C^0 e^{i t}+ C^{-0} e^{-i t}
 \end{equation*}
 \begin{equation*}
    u_2^0=-i r \left(  \frac{1}{T r k} - \frac{1}{2}\right) C^0 e^{+i t}+ i r \left(  \frac{1}{T r k} - \frac{1}{2}\right) C^{-0} e^{-i t}
 \end{equation*}
 Далее рассматриваем второе приближение:
\begin{equation}\label{su1}
       \frac{d u^1_1(t)}{d t}=T k (r \int_{-\infty}^{0} s e^{s}u^1_1(t+s)d s -  u^1_2 (t) ) + f^1_1 (t)
   \end{equation}
   \begin{equation*}
   \frac{d u^1_2(t)}{d t}= r(\frac{1}{T r k} -\frac{1}{2})u^1_1(t)+f^1_2 (t) ,
\end{equation*}
где $ f^2_1 $ и $f^2_2$ определяются формулами:
\begin{multline}
    f^1_1 (t)=T u_1^0 \left( r \int_{-\infty}^{0} s e^{s} y_1^1(t+s) d s - y_2^1 (t) \right)+T y_1^1 \left( r \int_{-\infty}^{0} s e^{s} u_1^0(t+s) d s -u_2^0 (t) \right)+\\+ T k r \int_{-\infty}^{0} s^2 e^{s} \lambda_1 u_1^0(t+s) d s - u_1^0 (t) \lambda_1 ,
    \end{multline}
    \begin{equation*}
    f^1_2(t)=\frac{1}{1-k}(\frac{1}{T r k} -\frac{1}{2}) \left(y_1^1 (t) u^0_2 (t) + u_1^0 (t) y^1_2 (t) \right) - u_2^0(t) \lambda_1.
    \end{equation*}
Полученные выражения можно разложить по резонансным гармоникам:
 \begin{equation}
      f^1_1(t)=d^1_1 e^{i t}+d^2_1 e^{2 i t} +d^{-1}_1e^{-i t}+d^{-2}_1 e^{-2 i t}
    \end{equation}
    \begin{equation*}
      f^1_2(t)=d^1_2 e^{i t}+d^2_2 e^{2 i t} +d^{-1}_2e^{-i t}+d^{-2}_2 e^{-2 i t},
    \end{equation*}
где коэффициенты разложения $ d^1_1 $, $ d^1_2 $, $ d^2_1$ и $d^2_2$ определяются формулами :
\begin{equation*}
    d^1_1=-\lambda_1 \left((1+i)\frac{T k r}{2} + 1\right) C^0,
\end{equation*}
\begin{equation*}
    d^{-1}_1=-\lambda_1 \left((1-i)\frac{T k r}{2} + 1\right) C^{-0},
\end{equation*}
\begin{equation*}
    d^2_1=\frac{2i}{k} C^0,
\end{equation*}
\begin{equation*}
    d^{-2}_1=\frac{-2i}{k} C^{-0},
\end{equation*}
\begin{equation*}
    d^1_2=\lambda_1 \left ( \frac{1}{T k r}-\frac{1}{2} \right) i r C^0,
\end{equation*}
\begin{equation*}
    d^{-1}_2=-\lambda_1 \left ( \frac{1}{T k r}-\frac{1}{2} \right) i r C^{-0},
\end{equation*}
\begin{equation*}
    d^2_2=\frac{-2i r}{1-k} \left ( \frac{1}{T k r}-\frac{1}{2} \right)^2 C^0
\end{equation*}
\begin{equation*}
    d^{-2}_2=\frac{2i r}{1-k} \left ( \frac{1}{T k r}-\frac{1}{2} \right)^2 C^{-0}
\end{equation*}
Условием существования 2$\pi$ -периодического решения является уравнение отбора:
\begin{equation*}
    \psi^{1*} d^1 =0,
\end{equation*}
\begin{equation*}
    \psi^{-1*} d^{-1} =0,
\end{equation*}
где $\psi$ - вектор решение сопряженной системы  :
 \begin{equation*}
     i \psi_1^1=T k r \psi_1^1 \frac{i}{2}- T k \psi_2^1
 \end{equation*}
 \begin{equation*}
     i \psi_2^1=r \left(  \frac{1}{T r k} - \frac{1}{2}\right) \psi_1^1 ,
 \end{equation*}
 Откуда получаем
 \begin{equation*}
     \psi_1^1 =1
 \end{equation*}
 \begin{equation*}
     \psi_2^1=-i r \left(  \frac{1}{T r k} - \frac{1}{2}\right)
 \end{equation*}
Условие существование 2$\pi$-периодического решения принимает вид:
\begin{equation}
    \psi_1^{1*} d^1_1+\psi_2^{1*} d^1_2 =0
\end{equation}
\begin{equation*}
    \psi_1^{-1*} d^{-1}_1+\psi_2^{-1*} d^{-1}_2 =0
\end{equation*}
Подставляя значение $\psi^1$,$d^1$,$\psi^{-1}$ и $d^{-1}$ получим систему алгебраических уравнений:
\begin{equation*}
  -\lambda_1 \left((1+i)\frac{T k r}{2} + 1\right) C^0 +i r \left(  \frac{1}{T r k} - \frac{1}{2}\right) (\lambda_1 \left ( \frac{1}{T k r}-\frac{1}{2} \right) i r C^0)  =0
\end{equation*}
\begin{equation*}
  -\lambda_1 \left((1-i)\frac{T k r}{2} + 1\right) C^{-0} -i r \left(  \frac{1}{T r k} - \frac{1}{2}\right) (-\lambda_1 \left ( \frac{1}{T k r}-\frac{1}{2} \right) i r C^{-0})  =0
\end{equation*}
Откуда получаем следующие алгебраические уравнения:
\begin{equation*}
  -\lambda_1 C^0 \left( (1+i)\frac{T k r}{2} + 1 +r^2 \left ( \frac{1}{T k r}-\frac{1}{2} \right)^2 \right)  =0
\end{equation*}
\begin{equation*}
    -\lambda_1 C^{-0} \left( (1-i)\frac{T k r}{2} + 1 +r^2 \left ( \frac{1}{T k r}-\frac{1}{2} \right)^2 \right)  =0
\end{equation*}
Так как начальное приближение мы считаем нетривиальным $C^0 \neq 0$, $C^{-0} \neq 0$ и выражение в скобках отлично от 0, то можно сделать вывод, что $\lambda_1 =0 $. \\
Для продолжения необходимо найти второй член разложения решения:
\begin{equation*}
     u^1_1= C_1^1 e^{2 i t}+ C_1^{-1} e^{-2 i t}
  \end{equation*}
  \begin{equation*}
     u^1_2= C_2^1 e^{2 i t}+ C_2^{-1} e^{-2 i t}
  \end{equation*}
  Постоянные $C_1^1$,$C_2^1$ определяются из системы алгебраических уравнений:
  \begin{equation*}
     2i C_1^1=T k r C_1^1( \frac{4i}{25}+\frac{3}{25})- T k C_2^1 + \frac{2i}{k} C^0
 \end{equation*}
 \begin{equation*}
     2i C_2^1=r \left(  \frac{1}{T r k} - \frac{1}{2}\right) C_1^1-\frac{2 i r}{1-k} \left(\frac{1}{T r k} - \frac{1}{2} \right)^2 C^0
 \end{equation*}
 Сопряженная система записывается и для $C_1^{-1}$,$C_2^{-1}$ :
 \begin{equation*}
     -2i C_1^{-1}=T k r C_1^{-1}( \frac{-4i}{25}+\frac{3}{25})- T k C_2^{-1} + \frac{-2i}{k} C^{-0}
 \end{equation*}
 \begin{equation*}
     -2i C_2^{-1}=r \left(  \frac{1}{T r k} - \frac{1}{2}\right) C_1^{-1}+\frac{2 i r}{1-k} \left(\frac{1}{T r k} - \frac{1}{2} \right)^2 C^{-0}
 \end{equation*}
 Откуда получаем следующие результаты:
\begin{equation*}
    C^1_1 =\frac{C^0}{3(1-k)T k r( T^2 k^2 r^2 + 12 T k r + 100)}[f_1 + i l_1]
\end{equation*}
\begin{equation*}
    C^1_2 =\frac{C^0}{3(1-k)T^2 k^2 r (T^2 k^2 r^2 + 12 T k r + 100)}[f_2+ i l_2]
\end{equation*}
где $f_1$,$l_1$,$f_2$,$l_2$ имеют следующий вид
\begin{equation*}
   f_1=400 r T -416 k r T +24 k r^2 T^2-8 k^2 r^2 T^2 -4 k^3 r^3 T^3 
\end{equation*}
\begin{equation*}
    l_1=-200 + 188 k r T - 32 k r^2 T^2-6 k^2 r^2 T^2 - 3 k^3 r^3 T^3
\end{equation*}
\begin{equation*}
    f_2=-400 +408 k r T - 16 k r^2 T^2 -92 k^2 r^2 T^2 +8 k^2 r^3 T^3 -6 k^3 r^3 T^3
\end{equation*}
\begin{equation*}
    l_2=-200 r T + 208 k r T +88 k r^2 T^2 -100 k^2 r^2 T^2 +6 k^2 r^3 T^3 - T64 r^4 k^4
\end{equation*}

Далее строим третье приближение
\begin{equation}\label{s3}
       \frac{d u^2_1(t)}{d t}=T k \left(r \int_{-\infty}^{0} s e^{s}u^2_1(t+s)d s -  u^2_2 (t) \right) + f^2_1 (t)
   \end{equation}
   \begin{equation*}
   \frac{d u^2_2(t)}{d t}= r\left(\frac{1}{T r k} -\frac{1}{2}\right)u^2_1(t)+f^2_2 (t) ,
\end{equation*}
где $ f^2_1 $ и $f^2_2$ определяются формулами:
\begin{multline}
    f^2_1 (t)=\alpha_2 T k \left( r \int_{-\infty}^{0} s e^{s} u^0_1 (t+s) d s - u^0_2 (t) \right)+\mu_2 k \left( r \int_{-\infty}^{0} s e^{s} u^0_1 (t+s) d s - u^0_2 (t) \right)\\ +T y^1_1 (t) \left( r \int_{-\infty}^{0} s e^{s} u^1_1 (t+s) d s - u^1_2 (t) \right)+T y^2_1(t)\left( r \int_{-\infty}^{0} s e^{s} u^0_1 (t+s) d s - u^0_2 (t) \right)  \\+T k r \left(  2  \alpha_2 \int_{-\infty}^{0} s e^{s} u^0_1 (t+s)d s +  \alpha_2 \int_{-\infty}^{0}s^2 e^{s} u^0_1(t+s) d s  
    +\lambda_2 \int_{-\infty}^{0}s^2 e^{s} u^0_1(t+s) d s  \right) - u^0_1 (t) \lambda_2
    \end{multline}
    \begin{multline*}
    f^2_2(t)=\frac{1}{ (1-k)}\left( \frac{1}{T k r} - \frac{1}{2} \right)( \alpha_2  u^0_1(t)r(1-k)+\frac{\mu_2}{T}u^0_1 (t)r(1-k) +\\+ y^1_1(t) u^1_2(t) + y^2_1(t) y^0_2 (t)+ u^0_1(t)y^2_2 (t) + u^1_1 (t)y^1_2 (t)   ) - u^0_2(t) \lambda_2.
    \end{multline*}
Полученные выражения можно разложить по резонансным гармоникам:
 \begin{equation}
      f^2_1(t)=d^1_1 e^{i t}+d^3_1 e^{3 i t} +d^{-1}_1e^{-i t}+d^{-3}_1 e^{-3 i t}
    \end{equation}
    \begin{equation*}
      f^2_2(t)=d^1_2 e^{i t}+d^3_2 e^{3 i t} +d^{-1}_2e^{-i t}+d^{-3}_2 e^{-3 i t},
    \end{equation*}
где коэффициенты разложения $ d^1_1 $ и $ d^1_2 $, определяются формулами :
\begin{equation*}
    d^1_1=\alpha_2 T k r C^0 \left ( -0.5+i(\frac{1}{2}+\frac{1}{T k r}) \right)+\frac{i C^0}{T} \mu_2 - \lambda_2 C^0 (\frac{T k r}{2} (1+i) +1)+d_1,
\end{equation*}
\begin{equation*}
    d^{-1}_1=\alpha_2 T k r C^{-0} \left ( -0.5-i(\frac{1}{2}+\frac{1}{T k r}) \right)-\frac{i C^{-0}}{T} \mu_2 - \lambda_2 C^{-0} (\frac{T k r}{2} (1-i) +1)+q_1,
\end{equation*}
\begin{equation*}
    d^1_2=\alpha_2 C^0 r \left ( \frac{1}{T k r}-\frac{1}{2} \right)+\frac{r}{T} \mu_2 C^0 \left ( \frac{1}{T k r}-\frac{1}{2} \right)+d_2 +\left ( \frac{1}{T k r}-\frac{1}{2} \right)i r C^0 \lambda_2,
\end{equation*}
\begin{equation*}
    d^{-1}_2=\alpha_2 C^{-0} r \left ( \frac{1}{T k r}-\frac{1}{2} \right)+\frac{r}{T} \mu_2 C^{-0} \left ( \frac{1}{T k r}-\frac{1}{2} \right)+q_2 -\left ( \frac{1}{T k r}-\frac{1}{2} \right)i r C^{-0} \lambda_2,
\end{equation*}
где 
\begin{multline*}
     d_1= C^0 (\frac{T r}{25} (3 F_1 -4 L_1 + i (4 F_1 +3 L_1))- T F_2 -i T L_2 + \frac{1}{k} L_1 - \frac{i}{k} F_1)+\\+C^{-0} (\frac{T r}{25} (3 A_1 -4 B_1 + i (4 A_1 +3 B_1))- T A_2 -i T B_2 + \frac{1}{k} B_1 - \frac{i}{k} A_1)
\end{multline*}
\begin{multline*}
     q_1= C^0 (\frac{T r}{25} (3 A_1 -4 B_1 - i (4 A_1 +3 B_1))- T A_2 +i T B_2 + \frac{1}{k} B_1 + \frac{i}{k} A_1)+\\+C^{-0} (\frac{T r}{25} (3 F_1 -4 L_1 - i (4 F_1 +3 L_1))- T F_2 +i T L_2 + \frac{1}{k} L_1 + \frac{i}{k} F_1)
\end{multline*}
\begin{multline*}
    d_2=\frac{1}{ (1-k)}\left( \frac{1}{T k r} - \frac{1}{2} \right) ( C^0 (F_2+i L_2 -r \left( \frac{1}{T k r} - \frac{1}{2} \right)(L_1-i F_1)+\\ + C^{-0}(A_2+i B_2 -r \left( \frac{1}{T k r} - \frac{1}{2} \right)(B_1-i A_1)))
\end{multline*}
\begin{multline*}
    q_2=\frac{1}{ (1-k)}\left( \frac{1}{T k r} - \frac{1}{2} \right) ( C^0 (A_2-i B_2 -r \left( \frac{1}{T k r} - \frac{1}{2} \right)(B_1+i A_1))+\\ + C^{-0}(F_2-i L_2 -r \left( \frac{1}{T k r} - \frac{1}{2} \right)(L_1+i F_1))
\end{multline*}
а $A_1$, $B_1$, $A_2$, $B_2$, $F_1$, $F_2$,$L_1$ и $L_2$ являются коэффициентами разложения вторых членов асимптотического разложения решения по гармоникам:
\begin{equation*}
    y_1^2(t)=(A_1+i B_1)e^{2 i t} + (A_1 -i B_1)e^{-2 i t}
\end{equation*}
\begin{equation*}
    y_2^2(t)=(A_2+i B_2)e^{2 i t} + (A_2 -i B_2)e^{-2 i t}
\end{equation*}
\begin{equation*}
    u_1^1(t)=(F_1+i L_1)e^{2 i t} + (F_1 -i L_1)e^{-2 i t}
\end{equation*}
\begin{equation*}
    u_2^1(t)=(F_2+i L_2)e^{2 i t} + (F_2 -i L_2)e^{-2 i t}
\end{equation*}
Условием существования 2$\pi$ -периодического решения является уравнение отбора:
\begin{equation*}
    \psi^{1*} d^1 =0,
\end{equation*}
\begin{equation*}
    \psi^{-1*} d^{-1} =0,
\end{equation*}
где $\psi$ - вектор решение сопряженной системы  :
 \begin{equation*}
     i \psi_1^1=T k r \psi_1^1 \frac{i}{2}- T k \psi_2^1
 \end{equation*}
 \begin{equation*}
     i \psi_2^1=r \left(  \frac{1}{T r k} - \frac{1}{2}\right) \psi_1^1 ,
 \end{equation*}
 Откуда получаем
 \begin{equation*}
     \psi_1^1 =1
 \end{equation*}
 \begin{equation*}
     \psi_2^1=-i r \left(  \frac{1}{T r k} - \frac{1}{2}\right)
 \end{equation*}
Условие существование 2$\pi$-периодического решения принимает вид:
\begin{equation}
    \psi_1^{1*} d^1_1+\psi_2^{1*} d^1_2 =0
\end{equation}
\begin{equation*}
    \psi_1^{-1*} d^{-1}_1+\psi_2^{-1*} d^{-1}_2 =0
\end{equation*}
Подставляя значение $\psi^1$,$d^1$,$\psi^{-1}$ и $d^{-1}$ получим систему алгебраических уравнений:
\begin{multline*}
  \alpha_2 T k r C^0 \left ( -0.5+i(\frac{1}{2}+\frac{1}{T k r}) \right)+\frac{i C^0}{T} \mu_2 - \lambda_2 C^0 (\frac{T k r}{2} (1+i) +1)+d_1 +i r \left(  \frac{1}{T r k} - \frac{1}{2}\right)\times\\ \times(\alpha_2 C^0 r \left ( \frac{1}{T k r}-\frac{1}{2} \right)+\frac{r}{T} \mu_2 C^0 \left ( \frac{1}{T k r}-\frac{1}{2} \right)+d_2 +\left ( \frac{1}{T k r}-\frac{1}{2} \right)i r C^0 \lambda_2)  =0
\end{multline*}
\begin{multline*}
 \alpha_2 T k r C^{-0} \left ( -0.5-i(\frac{1}{2}+\frac{1}{T k r}) \right)-\frac{i C^{-0}}{T} \mu_2 - \lambda_2 C^{-0} (\frac{T k r}{2} (1-i) +1)+q_1 -i r \left(  \frac{1}{T r k} - \frac{1}{2}\right)\times \\ \times (\alpha_2 C^{-0} r \left ( \frac{1}{T k r}-\frac{1}{2} \right)+\frac{r}{T} \mu_2 C^{-0} \left ( \frac{1}{T k r}-\frac{1}{2} \right)+q_2 -\left ( \frac{1}{T k r}-\frac{1}{2} \right)i r C^{-0} \lambda_2)  =0
\end{multline*}
Откуда получаем следующие алгебраические уравнения относительно постоянных $C^0$, $C^{-0}$:
\begin{equation}\label{opr}
  W_1 C^0 + E_1 C^{-0} =0
\end{equation}
\begin{equation*}
  W_2 C^0 + E_2 C^{-0} =0 ,
\end{equation*}
где коэффициенты $W_1$, $W_2$, $E_1$ и $E_2$ имеют следующий вид:
\begin{multline*}
W_1=\alpha_2 \left( T k r  \left ( -0.5+i(\frac{1}{2}+\frac{1}{T k r}) \right) +i(\frac{1}{2}+\frac{1}{T k r})^2 r^2 \right)+\frac{i \mu_2}{T} \left( 1+(\frac{1}{2}-\frac{1}{T k r})^2 r^2 \right)-\\  - \lambda_2 \left( \frac{T k r}{2}   (1+i)+1+(\frac{1}{2}-\frac{1}{T k r})^2 r^2 \right)+H_1+i J_1
\end{multline*}
\begin{multline*}
E_1=\frac{T r}{25} (3 A_1 - 4 B_1)-T A_2+\frac{1}{k} B_1 - (\frac{1}{2}-\frac{1}{T k r})^2 \frac{r}{1-k}B_2+ (\frac{1}{2}-\frac{1}{T k r})^3 \frac{r^2}{1-k}A_1+\\ + i \left( \frac{T r}{25} (3 B_1 +4 A_1)-T B_2-\frac{1}{k} A_1 - (\frac{1}{2}-\frac{1}{T k r})^2 \frac{r}{1-k}A_2+ (\frac{1}{2}-\frac{1}{T k r})^3 \frac{r^2}{1-k}B_1 
\right)
\end{multline*}
\begin{multline*}
W_2=\frac{T r}{25} (3 A_1 - 4 B_1)-T A_2+\frac{1}{k} B_1 - (\frac{1}{2}-\frac{1}{T k r})^2 \frac{r}{1-k}B_2+ (\frac{1}{2}-\frac{1}{T k r})^3 \frac{r^2}{1-k}A_1+\\ - i \left( \frac{T r}{25} (3 B_1 +4 A_1)-T B_2-\frac{1}{k} A_1 - (\frac{1}{2}-\frac{1}{T k r})^2 \frac{r}{1-k}A_2+ (\frac{1}{2}-\frac{1}{T k r})^3 \frac{r^2}{1-k}B_1 
\right)
\end{multline*}
\begin{multline*}
E_2=\alpha_2 \left( T k r  \left ( -0.5-i(\frac{1}{2}+\frac{1}{T k r}) \right) -i(\frac{1}{2}+\frac{1}{T k r})^2 r^2 \right)-\frac{i \mu_2}{T} \left( 1+(\frac{1}{2}-\frac{1}{T k r})^2 r^2 \right)-\\  - \lambda_2 \left( \frac{T k r}{2}   (1-i)+1+(\frac{1}{2}-\frac{1}{T k r})^2 r^2 \right)+H_1-i J_1,
\end{multline*}
где $H_1$, $J_1$ определяются формулами:
\begin{multline*}
H_1=\frac{T r}{25} (3F_1 - 4 L_1)-T F_2+\frac{1}{k}L_1 - (\frac{1}{2}-\frac{1}{T k r})^2 \frac{r}{1-k}L_2-+ (\frac{1}{2}-\frac{1}{T k r})^3 \frac{r^2}{1-k}F_1
\end{multline*}
\begin{multline*}
    J_1= \frac{T r}{25} (3 L_1 +4 F_1)-T L_2-\frac{1}{k} F_1 + (\frac{1}{2}-\frac{1}{T k r})^2 \frac{r}{1-k}F_2+ (\frac{1}{2}-\frac{1}{T k r})^3 \frac{r^2}{1-k}L_1 
\end{multline*}
Так как начальное приближение мы считаем нетривиальным $C^0 \neq 0$, $C^{-0} \neq 0$,то тогда определитель системы алгебраических уравнений (\ref{opr}) должен быть равен 0. Соответственно:
\begin{equation*}
    W_1 E_2 - E_1 W_2 =0
\end{equation*}
Откуда после алгебраических преобразований получаем квадратное уравнение для $\lambda_2$:
\begin{equation}
    \lambda_2 ^2 (a^2 + b^2) + \lambda_2 (2 \mu_2 a d - 2 b H_1 + 2 a J_1)=0,
\end{equation}
где $a$, $b$- находятся по формулам:
\begin{equation*}
    a=-\frac{T k r}{2}
\end{equation*}
\begin{equation*}
    b=1+\frac{T k r}{2} +(\frac{1}{2}- \frac{1}{T k r})^2 r^2
\end{equation*}
\begin{equation*}
    d=\frac{1}{T}(1 +(\frac{1}{2}- \frac{1}{T k r})^2 r^2)
\end{equation*}
Решением такого уравнения являются корни:
\begin{equation}
    \begin{cases}
    \lambda_2 =0 \\
    \lambda_2 = \frac{-p}{a^2+b^2}
    \end{cases}
\end{equation}
где $p=2 \mu_2 a d - 2 b H_1 + 2 a J_1$.\\
Нас интересует знак второго корня уравнения, который противоположен знаку p, так как знаменатель неотрицателен. Получаем, что если $p>0$, то $\lambda_2 <0$ и периодическое решение устойчиво, а если $p<0$, то $\lambda_2 >0$ и периодическое решение будет неустойчивым.
\begin{figure}
    \centering
    \includegraphics[scale=0.5]{l2area}
    \caption{Выделенные области соответствуют отрицательным значения параметра p }
    \label{fig:l2}
\end{figure}
Из (Рис. \ref{fig:l2}) мы видим, что есть области неустойчивости. Но они так же могут соответствовать областям $mu_2 <0$, где вообще говоря периодическое решение неопределено. \\
Введем новый параметр $p$ = $\frac{\lambda_2}{\mu_2}$, и будем рассматривать его знак относительно 0.
При численном исследовании на всей области определения параметров, с условием выполнения всех ограничений параметр p<0. Значит реализуются 2 случая:
\begin{enumerate}
 \item $ \mu_2 >0 $ и $\lambda_2 <0$, тогда периодические решения вне области устойчивости существуют и они устойчивы\\
 \item $ \mu_2 < 0 $ и $\lambda_2 >0$, тогда периодические решения есть внутри области устойчивости и они неустойчивы.\\
 \end{enumerate}
Через исходные параметры $\lambda_2$ определяется формулой:
\begin{equation*}
    \lambda_2 = \frac{q}{n} ,
\end{equation*}
где $q$, $n$ определяются формулами:
\begin{multline*}
   q = -0.3125 k^9 r^6 T^8-2.5 k^9 r^4 T^8-0.0625 k^8 r^7 T^7-0.5 k^8 r^5 T^7+5.5 k^8 r^4 T^8-38 k^8 r^3 T^7+\\+0.5 k^7 r^6 T^6+0.5 k^7 r^5 T^7-3 k^7 r^4 T^8-7.25 k^7 r^4 T^6+46 k^7 r^3 T^7-4 k^7 r^2 T^6+0.375 k^6 r^6 T^6-\\-7.75 k^6 r^5 T^5+6 k^6 r^4 T^6-8 k^6 r^3 T^7+108 k^6 r^3 T^5+72 k^6 r^2 T^6+88 k^6 r T^5+2.5 k^5 r^5 T^5+8 k^5 r^4 T^6+\\+65 k^5 r^4 T^4-80 k^5 r^3 T^5-68 k^5 r^2 T^6-327 k^5 r^2 T^4-88 k^5 r T^5-200 k^5 T^4-47.5 k^4 r^4 T^4-\\-32 k^4 r^3 T^5-255 k^4 r^3 T^3+232 k^4 r^2 T^4+344 k^4 r T^3+200 k^4 T^4+220 k^3 r^3 T^3+32 k^3 r^2 T^4+\\+512 k^3 r^2 T^2-200 k^3 r T^3-92 k^3 T^2-470 k^2 r^2 T^2-516 k^2 r T+488 k r T+208 k-200
\end{multline*}
\begin{equation*}
   n= 3(k-1)^2 k^6 r T^5 \left(k^2 r^2 T^2+12 k r T+100\right)\left(\frac{1}{4} k^2 r^2 T^2+\left(r^2 \left(\frac{1}{k r T}-\frac{1}{2}\right)^2+\frac{k r T}{2}+1\right)^2\right)
\end{equation*}

\newpage
\section*{ЗАКЛЮЧЕНИЕ}
\addcontentsline{toc}{section}{ЗАКЛЮЧЕНИЕ}
Натурные наблюдения показывают, что численности популяций млекопитающих в естественной среде колеблются с течением времени. С помощью специального выбора математической модели обеспечивают существование устойчивых колебаний. Опыт математического моделирования подтверждает важную роль запаздывания для реализации устойчивых колебательных процессов. В пределе T=0 модель Мэя \cite{has} переходит в систему Лотки-Вольтерры без временных запаздываний, которая обладает принципиальным недостатком – отсутствием изолированных периодических решений, которые обнаружены в ходе натурных наблюдений.
При изучении периодических колебаний в популяционной модели Мэя  мы использовали популярный метод Хопфа, позволяющий описывать бифуркацию рождения периодического решения из положения  равновесия \cite{has}. Численные эксперименты показывают, что периодические колебания в системе Мэя сохраняются с ростом параметра $\mu$.

\newpage
\begin{thebibliography}{5}
    \bibitem{Dolgiy}
    Ю.Ф. Долгий, П.Г.Сурков. Математические модели динамических систем с запаздыванием. Екатеринбург. Издательство Уральского университета.2012г.
    \bibitem{has}
    Хэссард Б., Казаринов Н., Вэн И. Теория и приложения бифуркации рождения цикла. М. : Мир, 1985.
    \bibitem{baz}
    Базыкин А.Д. Нелинейная динамика взаимодействующих популяций. Москва-Ижевск: Институт компьютерных исследований, 2003г.
    \bibitem{svir}
    Свирежев Ю.М., Логофет Д.О. Устойчивость биологических сообществ. Главная редакция физико-математической литературы издательства «Наука», М.,1978.
    \end{thebibliography}
\addcontentsline{toc}{section}{СПИСОК ЛИТЕРАТУРЫ}

\appendix

%\section*{ПРИЛОЖЕНИЯ}
%\addcontentsline{toc}{section}{ПРИЛОЖЕНИЯ}
\end{document}
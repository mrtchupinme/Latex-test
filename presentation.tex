\documentclass[12pt]{beamer}
\usepackage[russian]{babel}
\usepackage[utf8x]{inputenc}
\usepackage{amsmath}
\usepackage{graphicx}
\graphicspath{ {./images/} }
\title{Бифуркация Хопфа}
\author{Chupin Ilya}
\institute{Ural Federal University}
\date{2019}
\usetheme{Madrid}
\usecolortheme{beaver}
\begin{document}
\frame{\titlepage}
\begin{frame}
\frametitle{Бифуркация Хопфа}
 \begin{multline}
      y(t,\gamma)=\sum_{k=1}^n \gamma^k y^k (t)= \gamma y(t)+\gamma^2 y^2(t)+\gamma^3 y^3(t)+\gamma^4 y^4 (t)\\  + \gamma^5 y^5(t)+\gamma^6 y^6(t) +\gamma^7 y^7(t) +\gamma^8 y^8(t)+\gamma^9 y^9(t)
 \label{math/1}
  \end{multline}
Это уравнение (\ref{math/1}) является решением \emph{ первого приближения }
\end{frame}
\begin{frame}{Формулы}
    \frametitle{Наши формулы}
    \begin{columns}
    \column{0.5\textwidth}
   \small{ Это наша первая формула}\\
    \begin{multline}
    \small{  y(t,\gamma)=\sum_{k=1}^n \gamma^k y^k (t)}
 \label{math/2}
  \end{multline}
  \column{0.5\textwidth}
  Формула \ref{math/2} представляет собой степенной ряд нашего решения в разложении по вспомогательному параметру $\gamma $. Информацию о модели Мея вы можете найти на \cite{Dolgiy}[стр.~115].\\
\textbf{ Формулу (\ref{math/1}) можно встретить в таких учебных пособиях как \cite{Pimenov} и \cite{landau}.}
    \end{columns}
\end{frame}
\begin{frame}{Picture}
    \frametitle{Символ года}
    \begin{figure}
    \centering
    \includegraphics[scale=0.15, angle=45]{nhl}
    \caption{it is a good day to ...}
    \label{picture1}
\end{figure}
\end{frame}
\begin{frame}{biblio}
    \frametitle{Библиография}
    \begin{thebibliography}{5}
    \bibitem{Dolgiy}
    Ю.Ф. Долгий, П.Г.Сурков. Математические модели динамических систем с запаздыванием. Екатеринбург. Издательство Уральского университета.2012г.
    \bibitem{Pimenov}
    В.Г. Пименов. Численные методы. Екатеринбург. Издательство Уральского университета, 2012г.
    \bibitem{landau}
    Л.Д.Ландау, Е.М.Лифшиц. Статистическая физика.Москва, 1969г.
    \end{thebibliography}
    \end{frame}
\end{document}